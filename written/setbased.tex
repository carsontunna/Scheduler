\documentclass{article}

\begin{document}

\section{Set-based search}

\subsection{Model}


We define our model $A: (S,T)$ for use with a set-based search
paradigm. Given the set of \textit{TA}'s $t_i \in TA$ and
\textit{labs} $l_j \in LABS$, a fact is defined as a set of unique
2-tuples where the first element is a TA and the second element is a
lab. We further constrain the definition of a fact such that no
hard-constraints are violated. That is,

\begin{enumerate}
\item No TA has more than \textit{MAX\_LABS} labs: for any fact $f$,
  $\forall t_i |(t_i,l)\in f|\le MAX\_LABS$.
\item If a TA has a lab, that TA has at least \textit{MIN\_LABS} labs:
  for any fact $f$, $\forall t_i |(t_i,l)\in f|= 0 \lor |(t_i,l)\in
  f|\ge MIN\_LABS$
\item No lab has more than one TA and all labs have a TA: for any fact
  $f$, $\forall l_i |(t, l_i) \in f| = 1$
\item No TA has a time conflict: for any fact $f$, $\forall \left(
  t_i, l_i \right) \in f, time \left( l_i \right) \notin \lbrace time
  \left( c_k \right) \land c_k \in courses \left( t_i \right) \rbrace
  \land time(l_i) \notin \lbrace time(l_j) \land (t_i, l_j) \in f
  x\rbrace $ 

\end{enumerate}

The type $F$ is a set of facts and state $S$ is a superset of facts
($S = 2^F $). A transition $T:S\times S = \lbrace (s,s')| \exists A
\to B \in Ext • A \subseteq s \lor s' =(s-A) \cup B \rbrace$

Define $Ext : \lbrace A \to B | A,B \subseteq F \rbrace$ where $B = A
\cup C$ where $C$ is generated by specifying an allowable time and
calling \textit{Generate} then \textit{Combine} until that time is
exceeded or until $|C| = |A|$ so that $|B| = 2|A|$. The operations
used are defined as:

\begin{enumerate}

\item \textit{Generate} - Do a random walk through the defined in
  \ref{????}. The random walk does not compare leafs, it only tries
  paths at random until a solution is found. If no solution is found
  within the allowed time, this operation fails.

\item \textit{Combine} - First, map each element in a fact $f$ from a
  2-tuple to a 3-tuple $(t, l) \to (t, l, b = time(l))$. Then, for
  each $b$ such that $\exists (t',l',b) \in f$, match each instance of
  $t'$ and $l'$ once at random. This does not change the times that
  any TA teaches, it only changes which labs a TA is teaching. If the
  result violates any hard constraints, this operation fails. For the
  implementation, we will consider lazy evaluation of hard
  constraints.
\end{enumerate}

\subsection{Process}

We define our process $P: (A, Env, K)$ for the set based search. The
model $A$ has already been defined. It is assumed that the environment
$Env$ is unchanging so $K: S \times Env \to S$ is just $K: S \to S$.
The control $K$ is ??? from rubric ???.

$f_{wert}$ is defined as $-\sum\limits_{i}^{} penalty_i(f)$ where $
penalty_i$ is defined in Table \ref{????} as a function of a fact
which is either the penalty value from the table or zero if the
penalty does not apply.

$f_{select}$ is defined as a tournament. The number of facts is
``culled'' down to a specified number $N$. This is done by repeating
the following operation $|A|-N$ times. At random, two facts in $f_1,
f_2 \in F$ are selected. A random number $0 < r < 1$ is generated. If
$r < \frac{f_{1wert}}{f_{1wert} + f_{2wert}}$, $f_1$ is removed from
$A$, otherwise $f_2$ is removed from $A$. For the implementation,
$f_{wert}$ may not need to be calculated for all $f \in F$.

\end{document}
